\setcounter{tocdepth}{2}  %% Ueberschriften bis subsection ins Inhaltsverzeichnis
\setcounter{secnumdepth}{3}  %% Nummerierung bis subsection


%%% Codebeispiele - Style
\DeclareCaptionFont{white}{\color{white}}
\DeclareCaptionFormat{listing}{\colorbox{gray}{\parbox{13.5cm}{#1#2#3}}}
\captionsetup[lstlisting]{format=listing,labelfont=white,textfont=white}

% Entfernt Kapitel Ueberschrift
% Bsp.
% 	ALT:
%       Kapitel 1
%       Einführung
%
% 	NEU:
% 		1 Einführung
%
\renewcommand*\chapterheadstartvskip{\vspace{-\topskip}}

\begin{comment}
    % removes indentation of paragraphs
    \setlength{\parindent}{0pt}
\end{comment}

\begin{comment}
    % creates footnotes for the first use of an acronym
    \makeatletter
    \renewcommand{\ac}{\protect\@ac}%
    \renewcommand{\@ac}[1]{%
        \expandafter\ifx\csname ac@#1\endcsname\AC@used
            \acs{#1}%
        \else \acs{#1}\footnote{\acl{#1}}%
            \global\expandafter\let\csname ac@#1\endcsname\AC@used%
            \AC@addtoAC@clearlist{#1}%
            \AC@logged{#1}
        \fi
    }
    \makeatother
\end{comment}

